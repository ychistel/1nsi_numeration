\documentclass[11pt,a4paper]{article}

\usepackage{style2017}

\newcounter{numexo}
\begin{document}


\begin{NSI}
{Exercice}{Numération - Codage binaire}
\end{NSI}
\vspace{-0.5cm}
\addtocounter{numexo}{1}
\subsection*{\Large Exercice \thenumexo }
\begin{enumerate}
\item Donner l'écriture décimale des nombres entiers positifs codés en binaire sur un octet:\medskip

\begin{tabular}{*{4}{p{4.5cm}}}
\textbf{a)~$01010101$} & \textbf{b)~$10101010$} & \textbf{c)~$10111010$} & \textbf{d)~$11001100$}
\end{tabular}

\item Donner, en faisant bien apparaître la méthode, l'écriture en code binaire des entiers positifs suivants:\medskip

\begin{tabular}{*{4}{p{4.5cm}}}
\textbf{a)~$13$} & \textbf{b)~$54$} & \textbf{c)~$132$} & \textbf{d)~$245$}
\end{tabular}
\end{enumerate}


\addtocounter{numexo}{1}
\subsection*{\Large Exercice \thenumexo }
Un ordinateur calcule des additions, mais il le fait en binaire !
\begin{enumerate}
\item \begin{enumerate}
\item Calculer la somme, en binaire, des nombres $0101$ et $0110$.
\item Vérifier l'exactitude du résultat en passant en base 10.
\end{enumerate}
\item \begin{enumerate}
\item Calculer la somme, en binaire, des nombres $010011$ et $101101$.
\item Vérifier l'exactitude du résultat en passant en base 10.
\end{enumerate}
\end{enumerate}

\addtocounter{numexo}{1}
\subsection*{\Large Exercice \thenumexo }
Un ordinateur calcule aussi des multiplications en binaire. On s'intéresse à la multiplication par 2.
\begin{enumerate}
\item Donner l'écriture binaire de 8 et de son double.
\item Donner l'écriture binaire de 45 et de son double.
\item Recommencer avec un nombre de votre choix. 
\item Conjecturer une méthode simple de multiplication par 2.
\end{enumerate}

\addtocounter{numexo}{1}
\subsection*{\Large Exercice \thenumexo }
Les couleurs sont codées sur trois octets appelés composantes RGB. Le premier octet pour le rouge, le second pour le vert et le troisième pour le bleu. Les valeurs sont notées en décimal ou en hexadécimal.

Le mélange des trois composantes permet d'obtenir toute la palette de couleurs.
\begin{enumerate}
\item \begin{enumerate}
\item Combien de valeurs décimales peut-on avoir pour le rouge, le vert et le bleu ?
\item Combien de couleurs différentes peut-on disposer ?
\end{enumerate}
\item \begin{enumerate}
\item Le bleu acier a pour code RGB : (58, 142, 186). Donner une écriture hexadécimal de cette couleur.
\item La couleur orange a pour code hexadécimal ed7f10. Donner les valeurs décimales de chaque composante.
\end{enumerate}
\item \begin{enumerate}
\item Quel est le code du rouge en décimal et hexadécimal ? De même pour le vert et le bleu ?
\item Le blanc est obtenu en mélangeant rouge, vert et bleu. Quel est son code RGB en décimal et hexadécimal ?
\end{enumerate}
\item On appelle gris neutre une couleur dont les trois composantes RGB sont égales. Deux couleurs sont complémentaires lorsque par une synthèse additive on obtient un gris neutre.
\begin{enumerate}
\item Donner le code RGB de la couleur complémentaire du bleu.
\item Quelle est la couleur complémentaire du bleu ?
\end{enumerate}
\end{enumerate}

\end{document}